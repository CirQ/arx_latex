\section*{Appendix}

\renewcommand\thesubsection{\Alph{subsection}}

\subsection{Permutations by ARX}
\label{ssec:permarx}

The code below shows how to generate arbitrary permutations by solely ARX primitives.
This code is written in Python3 syntax, but is functionally equivalent to any other
programming languages.

\begin{figure}[h!]
\lstinputlisting[
    language=Python,
    %caption={Python program of permutations by ARX},
    %label={lst:perm-py}
]{permutation.py}
\vspace{-5mm}
\end{figure}



%\begin{figure}[t]
%\begin{center}
%\fbox{\rule{0pt}{2in} \rule{0.9\linewidth}{0pt}}
%   %\includegraphics[width=0.8\linewidth]{egfigure.eps}
%\end{center}
%   \caption{Example of caption.  It is set in Roman so that mathematics
%   (always set in Roman: $B \sin A = A \sin B$) may be included without an
%   ugly clash.}
%\label{fig:long}
%\label{fig:onecol}
%\end{figure}
%
%\begin{figure*}
%\begin{center}
%\fbox{\rule{0pt}{2in} \rule{.9\linewidth}{0pt}}
%\end{center}
%   \caption{Example of a short caption, which should be centered.}
%\label{fig:short}
%\end{figure*}
