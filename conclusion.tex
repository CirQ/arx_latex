\section{Conclusion}
\label{sec:conclusion}

This article introduces the ARX ciphers, which only consist of three primitives, modular
addition, intraword rotation and bitwise exclusive-or. ARX cipher is a kind of lightweight
ciphers hence is suitable for hardware that equipped with scarce computational resources.
The use of such lightweight ciphers also accelerate software performance, such as to
provide secure transmissions for IO-intensive channels. Given three primitives, it is shown
that ARX cipher can simulate all permutations (and functions) over $\mathbb{Z}_{2^n}$.
Finally, two instances, \textsc{Speck} and ChaCha are introduced. They jointly show the
practicability of ARX ciphers at both hardware and software levels.
