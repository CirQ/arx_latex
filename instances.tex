\section{Instances}
\label{sec:instances}

As mentioned before, the design of ARX ciphers ensures they are efficient at both
hardware and software levels. Hardware efficiency enables ARX can be run on most
constrained platforms, who typically equips with several KBs RAM and limited CPU power.
On the other hand, being efficient at software level also accelerate the PC applications
and services, for example, easing the burden of a busy Web server who accepts intensive
TLS connections.

In this section, two instances of ARX ciphers are introduced. The first one, called
\textsc{Speck}\cite{beaulieu2015simon2}, is designated on IoT devices. The second one,
called ChaCha\cite{bernstein2008chacha}, is a general-purposed stream cipher family,
and it is recently selected as the replacement of RC4 in the
TLS\footnote{\url{https://tools.ietf.org/html/rfc7905}} standard.

\subsection{The \textsc{\textbf{Speck}} Block Cipher}

\textsc{Speck} is designed specifically for encrypting data on constrained platforms.
It is also able to offer security services on diverse platforms with different security
levels, which are controlled by two security parameters.

\textsc{Speck} block cipher adopts the notation \textsc{Speck}$2n/nm$, where $2n$ is the
block size and $mn$ is the key size (in bits). In the design, $n\in\{16,24,32,48,64\}$
and each $n$ corresponds to a set of options for $m$. The \textsc{Speck} round function
is the (Feistel-based) map
$$R_k(x,y)=((\mathrm{r}^{-\alpha}x+y)\oplus k,\ \mathrm{r}^\beta y\oplus(\mathrm{r}^{-\alpha}x+y)\oplus k)$$
where $x$ and $y$ are $n$-bits quantities, and $k$ is the round key. The rotation
parameters $\alpha$ and $\beta$, and the number of rounds are specified along with
block size $n$.

\textsc{Speck} with $m$-size words accepts the initial key $K=(\ell_{m-2},\dots,\ell_0,k_0)$,
it then generates round keys as follows:

\begin{equation*}
\setlength{\abovedisplayshortskip}{0em}
\setlength{\abovedisplayskip}{0em}
\begin{split}
\ell_{i+m-1} &= (k_i+\mathrm{r}^{-\alpha}\ell_i)\oplus i\\
k_{i+1} &= \mathrm{r}^\beta k_i \oplus \ell_{i+m-1}
\end{split}
\end{equation*}

\noindent The value $k_i$ is the $i$-th round key.

\textsc{Speck}, by designed, uses \textit{simple} round function with necessary times
of rounds for security, as it offers compact implementation hence is suited for
constrained platforms. The use of uniform parameters (across different security levels)
also approves this purpose. Moreover, \textsc{Speck} can be done entirely in-place, so
unnecessary moves of word can be avoided to obtain better performance.

\subsection{The ChaCha Stream Cipher}

The cryptosystem ChaCha is a successor of the stream cipher Salsa20
\cite{bernstein2008salsa20}, which is designed to improve diffusion at each round. By
design, ChaCha is faster than existing standard ciphers, and is aimed at users who
value speed more than secrecy. ChaCha offers 3 variants, they are differed in the number
of rounds: 8, 12 and 20 rounds versions.

ChaCha itself computes a quarter-round to update 4 32-bits state words $(a,b,c,d)$ as:

\begin{equation*}
\setlength{\abovedisplayshortskip}{0em}
\setlength{\abovedisplayskip}{0em}
\begin{split}
a &= a+b\\[1pt]
c &= c+d\\[1pt]
a &= a+b\\[1pt]
c &= c+d\\[1pt]
\end{split}
\quad
\begin{split}
d &= \mathrm{r}^{16}(d\oplus a)\\
b &= \mathrm{r}^{12}(b\oplus c)\\
d &= \mathrm{r}^{8}(d\oplus a)\\
b &= \mathrm{r}^{7}(b\oplus c)\\
\end{split}
\end{equation*}

A ChaCha quarter-round gives each input word a chance to affect each output word, hence
achieves efficient diffusion at each round: every 1-bit input difference changes 12.5
output bits on average. This is the major difference between ChaCha and its predecessor
Salsa20, who affects on 8 bits per round.

ChaCha adopts the strategy that the plaintext and ciphertext do not affect the key stream,
in other word, it is a synchronous stream cipher. To apply the stream, ChaCha simply xor
it to the plaintext, as how a pseudorandom-OTP will work.

Google proposed ChaCha20 along with Poly1305 MAC as a replacement for RC4 in TLS, to
secures TLS/SSL traffic between the Chrome browser on Android and Google's websites.
Shortly after Google, both the ChaCha20 and Poly1305 were implemented for a new
cipher in OpenSSH. These adoptions imply that the ARX cipher family does not only
essential for hardware efficiency, but also important for software performance.
