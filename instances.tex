\section{Instances}
\label{sec:instances}

As mentioned before, the design of ARX ciphers ensures they are efficient at both
hardware and software levels. Hardware efficiency enables ARX can be run on most
constrained platforms, who typically equips with several KBs RAM and limited CPU power.
On the other hand, being efficient at software level also accelerate the PC applications
and services, for example, easing the burden of a busy Web server who accepts intensive
TLS connections.

In this section, two instances of ARX ciphers are introduced. The first one, called
\textsc{Speck}\cite{beaulieu2015simon2}, is designated on IoT devices. The second one,
called \textsc{ChaCha}\cite{bernstein2008chacha}, is a general-purposed stream cipher
family, and it is recently selected as the replacement of RC4 in the
TLS\footnote{\url{https://tools.ietf.org/html/rfc7905}} standard.
g
\subsection{The \textsc{\textbf{Speck}} Block Cipher}

\textsc{Speck} is designed specifically for encrypting data on constrained platforms.
It is also able to offer security services on diverse platforms with different security
levels, which are controlled by two security parameters.

\textsc{Speck} block cipher adopts the notation \textsc{Speck}$2n/nm$, where $2n$ is the
block size and $mn$ is the key size (in bits). In the design, $n\in\{16,24,32,48,64\}$
and each $n$ corresponds to a set of options for $m$. The \textsc{Speck} round function
is the (Feistel-based) map
$$R_k(x,y)=((\mathrm{r}^{-\alpha}x+y)\oplus k,\ \mathrm{r}^\beta y\oplus(\mathrm{r}^{-\alpha}x+y)\oplus k)$$
where $x$ and $y$ are $n$-bits quantities, and $k$ is the round key. The rotation
parameters $\alpha$ and $\beta$, and the number of rounds are specified along with
block size $n$.

\textsc{Speck} with $m$-size words accepts the initial key $K=(\ell_{m-2},\dots,\ell_0,k_0)$,
it then generates round keys as follows:

\begin{equation*}
\setlength{\abovedisplayshortskip}{0em}
\setlength{\abovedisplayskip}{0em}
\begin{split}
\ell_{i+m-1} &= (k_i+\mathrm{r}^{-\alpha}\ell_i)\oplus i\\
k_{i+1} &= \mathrm{r}^\beta k_i \oplus \ell_{i+m-1}
\end{split}
\end{equation*}

\noindent The value $k_i$ is the $i$-th round key.

\textsc{Speck}, by designed, uses \textit{simple} round function with necessary times
of rounds for security, as it offers compact implementation hence is suited for
constrained platforms. The use of uniform parameters (across different security levels)
also approves this purpose. Moreover, \textsc{Speck} can be done entirely in-place, so
unnecessary moves of word can be avoided to obtain better performance.

\subsection{The \textbf{\textsc{ChaCha}} Stream Cipher}

\lipsum[3]
