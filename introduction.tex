\section{Introduction}

All that cryptography concerns are the \textit{privacy} and the \textit{integrity}
of data, no matter where the data are being processed. During the past decades,
numerous cryptographic schemes are proposed to realize this two goals. However,
most of these schemes discount their application scenarios. An typical example is
that, as pointed out by Ray \etal \cite{beaulieu2015simon}, an RFID authentication
protocol may require only 64-bits secrecy, while AES providing at least 128-bits
is certainly a waste of chip space.

Recently, \textit{pervasive computing} and its applications enable us to do our daily
activity more efficiently. The applications of pervasive computing are built from highly
constrained devices, such as sensor networks and chip cards \cite{hansmann2013pervasive}.
These systems highly emphasize on the energy consumption of the operations that taken,
as well as the computation resource usage \cite{satyanarayanan2001pervasive}. Obviously,
the secure assurance should also be bound to such constrains. Given the fact that
existing block cipher standard (\ie, AES) is unsuitable for these constrained
environments, the need for \textit{lightweight cryptography} attracts more attentions.

ARX (originally termed as AXR \cite{weinmann2009axr}) describes a family of lightweight
cryptographic schemes, which claim their lightness and universality in terms of limited
lightweight operations. In particular, an instance of ARX cipher consists of only three
kinds of operations (or \textit{primitives}): modular addition, circular bit-shifting
(the rotation) and bitwise xor. These primitives can be efficiently supported by
individual machine instructions, with each of them runs typically in one clock cycle.
The simplicity enables their uses on most constrained computing environments, especially
on those energy-aware systems, \eg, wireless sensor networks \cite{perrig2004security}.

The property of efficiency is not only necessary for hardware support, but also for
software implementation. Actually, an efficient encryption scheme is also essential for
those IO-intensive applications over a secure channel. For example, efficient encryption
and authentication improve the throughput of a popular website that transmits data
over TLS protocol. ARX cipher can also accelerate software level security-services,
because its operations can be easily implemented by almost all high-level programming
languages.

Despite the efficiency concerns, ARX should be provably secure under well-known attacks
towards a cryptosystem. Showing that each primitive runs in constant time, ARX is robust
against timing side-channels \cite{beaulieu2015simon}. An ARX cipher gets its
nonlinearity solely from modular addition, while the other two operations provide only
linear components. Given such inherited limitation, ARX ciphers could be vulnerable
under rotational cryptanalysis \cite{khovratovich2010rotational}. Excepting these
well-known results, the security analysis on ARX ciphers is still a challenging problem
\cite{mouha2011arx}.

This article explores the ARX cipher family on several aspects. Specifically, Section
\ref{sec:definition} gives the formal definition of an ARX cipher, in terms of the
operations, the notations, and their properties. Section \ref{sec:expressiveness}
discusses the expressiveness of ARX ciphers, particularly, it's shown that any permutation
on $\mathbb{Z}_{2^n}$ can be solely represented by ARX operations. Section \ref{sec:instances}
introduces two instances of ARX cipher, with their constructions and applications.
Section \ref{sec:conclusion} concludes this article.
