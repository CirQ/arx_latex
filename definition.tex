\section{Definition}
\label{sec:definition}

ARX describes a family of lightweight cryptographic schemes. In particular, an
instance of ARX cipher consists of only three kinds of primitives: modular addition,
circular bit-shifting and bitwise xor. Formally, the ARX cipher family is defined
as follow:

\begin{definition}
A scheme $\Pi=(\mathit{Gen},\mathit{Enc},\mathit{Dec})$, with plaintext
space and ciphertext space $\mathcal{P}=\mathcal{C}\subseteq\mathbb{Z}_{2^n}^*$, is an
ARX cipher of security parameter $n$, if and only if all the three functions are
composites of the following three primitives:
\begin{itemize}
\setlength\partopsep{0em}
\setlength\topsep{0em}
\setlength\itemsep{0em}
\setlength\parskip{0em}
\item {\bf Addition}: binary operator of modular addition on $\mathbb{Z}_{2^n}$,
whose operands $x,y\in\mathbb{Z}_{2^n}$, denoted by $x+y$
\item {\bf Rotation}: intraword \textbf{left} rotation by $b$ bits to a quantity $x$
such that $b\in\mathbb{Z}_n$ and $x\in\mathbb{Z}_{2^n}$, denoted by $\mathrm{r}^bx$
\item {\bf Xor}: bitwise xor operation on $\mathbb{Z}_{2^n}$, denoted by $x\oplus y$
\end{itemize}
Denote this scheme by $\Pi\in\textit{ARX}(n)$. When there are no confusions, we can
also say the function $\mathit{Enc}$ and $\mathit{Dec}$ are $\textit{ARX}(n)$, denote
by $\mathit{Enc},\mathit{Dec}\in\textit{ARX}(n)$.
\end{definition}

The definition above does not distinguish the operands for addition and xor. In other
word, for operation $x+y$, $x$ can either be a word in the ciphertext, or a constant,
the same as $y$. The operands for $\oplus$ can also be ciphers or constants. As for
the operation $\mathrm{r}^bx$, the shift offset $b$ must be a constant in $\mathbb{Z}_n$.

The security parameter $n$ here does not correspond to the key size or block size of
an encryption scheme. Typically a block cipher involves a key scheduling function
which derives the round keys for each iteration from a fixed-sized key (the
\textit{seed}). The value of $n$ specifies how the round keys are calculated regarding
the quantity's word length, while the key size is the length of the seed. The block
size can also distinct from $n$, as DES (the Feistel construction) encrypts 64-bits
cipher blocks while the round function is a permutation on 32-bits strings.

As most lightweight ciphers adopt S-boxes as the nonlinear components, though ARX does
not provide such component. ARX obtains its nonlinearity solely from the modular
additions, hence may be vulnerable to rotational cryptanalysis. The trade-off here is,
ARX has more compact implementation, while S-box generally consumes more memory and
persistent storage. This property means ARX is suitable for most constrained platforms,
which are designed to equip with scarce resources. For example, a processing unit in a
wireless sensor network typically has several KBs memory for code and data, while their
OS code occupies nearly half of the available space \cite{akyildiz2002wireless}.

ARX can be realized by in-place operations, while each one can be run in constant time.
These two properties bring high performance to ARX ciphers on both software and hardware
levels. Table \ref{tab:arx} shows how these operations can be realized by in-place
Python statement and single MIPS32 instruction.

\begin{table*}[t]
\caption{Implementing in-place ARX operations by Python statements and MIPS32 instructions}
\label{tab:arx}
\medskip
\begin{center}
\begin{tabular}{c|l|l|l}
\hline
             & notation & Python statement & MIPS32 instruction \\ \hline
\textbf{Add} & $x=x+y$ & \texttt{x = (x + y) \% 2**n} & \texttt{add \$x, \$x, \$y} \\
\textbf{Rot} & $x=\mathrm{r}^bx$ & \texttt{x = ((x>>(n-b))|(x<<b)) \% 2**n} & \texttt{rol \$x, \$x, \#b} \\
\textbf{Xor} & $x=x\oplus y$ & \texttt{x = x \textasciicircum\ y} & \texttt{xor \$x, \$x, \$y} \\ \hline
\end{tabular}
\end{center}
\end{table*}

Another important note is that, running single operation in constant time mitigates
the possibility from suffering timing side-channel attacks. The abandon of lookup tables
(such as S-box) also prevents the system from \textit{cache-timing attacks}
\cite{bonneau2006cache}. The properties of ARX cipher show that that it is suitable on
most constrained computing environments, where the resources are limited but the secrecy
is still essential for various tasks. Also, it is efficient at software level hence
it is able to provide secure transmission for IO-intensive channels. Later in Section
\ref{sec:instances}, two such instances of ARX ciphers will be introduced.
