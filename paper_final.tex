\documentclass[10pt,twocolumn,letterpaper]{article}

\usepackage{cvpr}
\usepackage{times}
\usepackage{epsfig}
\usepackage{graphicx}
\usepackage{amsmath}
\usepackage{amssymb}

% Include other packages here, before hyperref.
\usepackage{lipsum}
\usepackage[utf8]{inputenc}
\usepackage[english]{babel}
\usepackage{amsthm}
\usepackage{enumitem}
\usepackage{mathrsfs}
\usepackage{multicol}
\usepackage[nottdefault]{sourcecodepro}
\usepackage{listings}

\newtheorem{theorem}{Theorem}
\newtheorem{corollary}{Corollary}
\newtheorem{lemma}{Lemma}

\theoremstyle{definition}
\newtheorem{definition}{Definition}

\lstset{
    breaklines=true,
    frame=lines,
    basicstyle=\fontfamily{SourceCodePro-TLF}\scriptsize,
    captionpos=b,
    prebreak=\raisebox{0ex}[0ex][0ex]{\ensuremath{\hookleftarrow}}
}


% If you comment hyperref and then uncomment it, you should delete
% egpaper.aux before re-running latex. (Or just hit 'q' on the first latex
% run, let it finish, and you should be clear).
\usepackage[breaklinks=true,bookmarks=false]{hyperref}

\cvprfinalcopy % *** Uncomment this line for the final submission
\def\cvprPaperID{****} % *** Enter the CVPR Paper ID here
\def\httilde{\mbox{\tt\raisebox{-.5ex}{\symbol{126}}}}

\setcounter{page}{1}
\begin{document}

\title{An Exploratory Study on ARX Cipher Family}

\author{
    Sinan WANG (11849180)\\
    {\tt\small 11849180@mail.sustech.edu.cn}
}

\maketitle

\begin{abstract}
Modern cryptography concerns the privacy and the integrity of data. However, most
cryptographic schemes fail to target on resources-constrained platforms, and that's
why \textbf{lightweight cryptography} is essential for these platforms. ARX ciphers
are such kind of lightweight cryptographic scheme. Given only three primitives,
modular Addition, intraword Rotation, and bitwise Xor, ARX ciphers are superior in
both security and efficiency. This article introduces ARX cipher family, in terms
of the formal definition, properties, and a brief security analysis. Particularly,
it is shown that ARX cipher can simulate any permutation (actually, all functions),
hence is sufficiently expressive. Two instances of ARX cipher are introduced, who
target on hardware implementation and software efficiency, correspondingly.
% rotational cryptanalysis?
\end{abstract}


\section{Introduction}

All that cryptography concerns are the \textit{privacy} and the \textit{integrity}
of data, no matter where the data are being processed. During the past decades,
numerous cryptographic schemes are proposed to realize this two goals. However,
most of these schemes discount their application scenarios. An typical example is
that, as pointed out by Ray \etal \cite{beaulieu2015simon}, an RFID authentication
protocol may require only 64-bits secrecy, while AES providing at least 128-bits
is certainly a waste of chip space.

Recently, \textit{pervasive computing} and its applications enable us to do our daily
activity more efficiently. The applications of pervasive computing are built from highly
constrained devices, such as sensor networks and chip cards \cite{hansmann2013pervasive}.
These systems highly emphasize on the energy consumption of the operations that taken,
as well as the computation resource usage \cite{satyanarayanan2001pervasive}. Obviously,
the secure assurance should also be bound to such constrains. Given the fact that
existing block cipher standard (\ie, AES) is unsuitable for these constrained
environments, the need for \textit{lightweight cryptography} attracts more attentions.

ARX (originally termed as AXR \cite{weinmann2009axr}) describes a family of lightweight
cryptographic schemes, which claim their lightness and universality in terms of limited
lightweight operations. In particular, an instance of ARX cipher consists of only three
kinds of operations (or \textit{primitives}): modular addition, circular bit-shifting
(the rotation) and bitwise xor. These primitives can be efficiently supported by
individual machine instructions, with each of them runs typically in one clock cycle.
The simplicity enables their uses on most constrained computing environments, especially
on those energy-aware systems, \eg, wireless sensor networks \cite{perrig2004security}.

The property of efficiency is not only necessary for hardware support, but also for
software implementation. Actually, an efficient encryption scheme is also essential for
those IO-intensive applications over a secure channel. For example, efficient encryption
and authentication improve the throughput of a popular website that transmits data
over TLS protocol. ARX cipher can also accelerate software level security-services,
because its operations can be easily implemented by almost all high-level programming
languages.

Despite the efficiency concerns, ARX should be provably secure under well-known attacks
towards a cryptosystem. Showing that each primitive runs in constant time, ARX is robust
against timing side-channels \cite{beaulieu2015simon}. An ARX cipher gets its
nonlinearity solely from modular addition, while the other two operations provide only
linear components. Given such inherited limitation, ARX ciphers could be vulnerable
under rotational cryptanalysis \cite{khovratovich2010rotational}. Excepting these
well-known results, the security analysis on ARX ciphers is still a challenging problem
\cite{mouha2011arx}.

This article explores the ARX cipher family on several aspects. Specifically, Section
\ref{sec:definition} gives the formal definition of an ARX cipher, in terms of the
operations, the notations, and their properties. Section \ref{sec:expressiveness}
discusses the expressiveness of ARX ciphers, particularly, it's shown that any permutation
on $\mathbb{Z}_{2^n}$ can be solely represented by ARX operations. Section \ref{sec:instances}
introduces two instances of ARX cipher, with their constructions and applications.
Section \ref{sec:conclusion} concludes this article.


\section{Definition}
\label{sec:definition}

ARX describes a family of lightweight cryptographic schemes. In particular, an
instance of ARX cipher consists of only three kinds of primitives: modular addition,
circular bit-shifting and bitwise xor. Formally, the ARX cipher family is defined
as follow:

\begin{definition}
A scheme $\Pi=(\mathit{Gen},\mathit{Enc},\mathit{Dec})$, with plaintext
space and ciphertext space $\mathcal{P}=\mathcal{C}\subseteq\mathbb{Z}_{2^n}^*$, is an
ARX cipher of security parameter $n$, if and only if all the three functions are
composites of the following three primitives:
\begin{itemize}
\setlength\partopsep{0em}
\setlength\topsep{0em}
\setlength\itemsep{0em}
\setlength\parskip{0em}
\item {\bf Addition}: binary operator of modular addition on $\mathbb{Z}_{2^n}$,
whose operands $x,y\in\mathbb{Z}_{2^n}$, denoted by $x+y$
\item {\bf Rotation}: intraword \textbf{left} rotation by $b$ bits to a quantity $x$
such that $b\in\mathbb{Z}_n$ and $x\in\mathbb{Z}_{2^n}$, denoted by $\mathrm{r}^bx$
\item {\bf Xor}: bitwise xor operation on $\mathbb{Z}_{2^n}$, denoted by $x\oplus y$
\end{itemize}
Denote this scheme by $\Pi\in\textit{ARX}(n)$. When there are no confusions, we can
also say the function $\mathit{Enc}$ and $\mathit{Dec}$ are $\textit{ARX}(n)$, denote
by $\mathit{Enc},\mathit{Dec}\in\textit{ARX}(n)$.
\end{definition}

The definition above does not distinguish the operands for addition and xor. In other
word, for operation $x+y$, $x$ can either be a word in the ciphertext, or a constant,
the same as $y$. The operands for $\oplus$ can also be ciphers or constants. As for
the operation $\mathrm{r}^bx$, the shift offset $b$ must be a constant in $\mathbb{Z}_n$.

The security parameter $n$ here does not correspond to the key size or block size of
an encryption scheme. Typically a block cipher involves a key scheduling function
which derives the round keys for each iteration from a fixed-sized key (the
\textit{seed}). The value of $n$ specifies how the round keys are calculated regarding
the quantity's word length, while the key size is the length of the seed. The block
size can also distinct from $n$, as DES (the Feistel construction) encrypts 64-bits
cipher blocks while the round function is a permutation on 32-bits strings.

As most lightweight ciphers adopt S-boxes as the nonlinear components, though ARX does
not provide such component. ARX obtains its nonlinearity solely from the modular
additions, hence may be vulnerable to rotational cryptanalysis. The trade-off here is,
ARX has more compact implementation, while S-box generally consumes more memory and
persistent storage. This property means ARX is suitable for most constrained platforms,
which are designed to equip with scarce resources. For example, a processing unit in a
wireless sensor network typically has several KBs memory for code and data, while their
OS code occupies nearly half of the available space \cite{akyildiz2002wireless}.

ARX can be realized by in-place operations, while each one can be run in constant time.
These two properties bring high performance to ARX ciphers on both software and hardware
levels. Table \ref{tab:arx} shows how these operations can be realized by in-place
Python statement and single MIPS32 instruction.

\begin{table*}[t]
\caption{Implementing in-place ARX operations by Python statements and MIPS32 instructions}
\label{tab:arx}
\medskip
\begin{center}
\begin{tabular}{c|l|l|l}
\hline
             & notation & Python statement & MIPS32 instruction \\ \hline
\textbf{Add} & $x=x+y$ & \texttt{x = (x + y) \% 2**n} & \texttt{add \$x, \$x, \$y} \\
\textbf{Rot} & $x=\mathrm{r}^bx$ & \texttt{x = ((x>>(n-b))|(x<<b)) \% 2**n} & \texttt{rol \$x, \$x, \#b} \\
\textbf{Xor} & $x=x\oplus y$ & \texttt{x = x \textasciicircum\ y} & \texttt{xor \$x, \$x, \$y} \\ \hline
\end{tabular}
\end{center}
\end{table*}

Another important note is that, running single operation in constant time mitigates
the possibility from suffering timing side-channel attacks. The abandon of lookup tables
(such as S-box) also prevents the system from \textit{cache-timing attacks}
\cite{bonneau2006cache}. The properties of ARX cipher show that that it is suitable on
most constrained computing environments, where the resources are limited but the secrecy
is still essential for various tasks. Also, it is efficient at software level hence
it is able to provide secure transmission for IO-intensive channels. Later in Section
\ref{sec:instances}, two such instances of ARX ciphers will be introduced.


\section{Expressiveness}
\label{sec:expressiveness}

A major advantage of ARX cipher family is the expressiveness
\footnote{The content is inspired by a question on StackExchange:
\url{https://crypto.stackexchange.com/questions/51412/}}.
With only three primitives, ARX cipher is able to perform almost all confusion and
diffusion techniques. Theoretically, it is shown that $\textit{ARX}(n)$ primitives
can simulate all permutations over $\mathbb{Z}_{2^n}$, hence replace the S-box in
block cipher constructions. A more general conclusion states that, $\textit{ARX}(n)$
can represent all functions $\mathbb{Z}_{2^n}^m\rightarrow\mathbb{Z}_{2^n}$ for any
positive integer $m$.

\subsection{Simulating Permutations}

A \textit{permutation} of the finite set $X$ is a bijective function $\pi:\textit{X}\rightarrow X$.
Such $\pi$ is called a permutation over $X$. In other word, permutation is the
rearrangement of a finite sequence. It is trivial to see that all permutations can be
generated by solely swapping operations on any other permutation. The swapping on a
finite sequence is defined as follow:

\begin{definition}
Let sequence $X=(x_1,x_2,\cdots,x_n)$, \textit{swapping} two elements $x_i$, $x_j$ in
$X$ resulting another sequence $X'=(x'_1,x'_2,\cdots,x'_n)$ such that:
\begin{align*}
x_i &= x'_j \\
x_j &= x'_i \\
x_k &= x'_k\ \mathrm{for}\ k\notin\{i,j\}
\end{align*}
\end{definition}

Many block ciphers adopt permutations as the nonlinear components, which also called
the \textit{S-boxes} \cite{stinson2005cryptography}. ARX cipher obtains its nonlinearity
from the modular addition, which is able to form (or \textit{simulate}) more complicated
nonlinear components. A nice property of ARX family states that, it can simulate all
permutations over the message space:

\begin{theorem}
\label{thm:perm}
For all permutations $\pi$ over $\mathbb{Z}_{2^n}$, $\pi\in\textit{ARX}(n)$.
\end{theorem}

Given arbitrary permutation, say that, the permutation in natural ordering, it is able to
generate any desired permutation by a pile of swapping operations on specific pairs of
permutation elements. Then, by showing swapping can be realized by ARX operations,
Theorem \ref{thm:perm} can be proven:

\begin{proof}
We proof this by construction, that is, we will construct a function $\sigma$ such that
$\sigma\textit{(X)}$ swaps two particular elements in the permutation $X$. Note that
$\sigma$ doesn't consider the positions of the elements being swapped, however, since all
elements in a permutation are unique, this definition does not introduce any ambiguity.

The problem can therefore be simplified as, constructing an ARX function $\sigma$ such
that, for input pair $(a,b)$, there are $\sigma(a)\textit{=}b$, $\sigma(b)\textit{=}a$,
and $\sigma(x)\textit{=}x$ for $x\notin\{a,b\}$. Denote this function by $f_{a,b}(x)$.
Given multiple such functions, their composition is thus a permutation over the domain.

Without loss of generality, let $a\neq b$. For simplicity, let $x-y=x+(2^n-y)$, and
$\mathrm{r}^{-b}(x)=\mathrm{r}^{n-b}(x)$ (the right rotation). We then construct
$f_{a,b}(x)$ by the following routine:

\begin{enumerate}
\setlength\partopsep{0em}
\setlength\topsep{0em}
\setlength\itemsep{0em}
\setlength\parskip{0em}
\item let $a'=0$, $b'=b-a$, the base function $f_0(x)\textit{=}x-a$
\item find the integer $i$ such that $\mathrm{r}^{-i}(b')$ is an odd number, set
$b'=\mathrm{r}^{-i}(b')$ and function $f_1(x)\textit{=}\mathrm{r}^{-i}(f_0(x))$, denoted
by $f_1\textit{=}\mathrm{r}^{-i}(f_0)$
\item \label{enum:loopcond} loop Step \ref{enum:loop} for $b'\neq1$, otherwise, go to
Step \ref{enum:loopout}
\item \label{enum:loop} let $b'=(b'\oplus1)-1$, function $f_i\textit{=}(f_{i-1}\oplus1)-1$
\item \label{enum:loopout} let $f_k=f_{i}+2^n-2$, and the auxiliary function
$g(x)=\mathrm{r}^1(\mathrm{r}^{-1}(x+2)-1)$
\item output function $\sigma=f_k^{-1}(g(f_k))$ as $f_{a,b}$
\end{enumerate}

\noindent Here we take $f_i$ as the last function when the looping condition at Step
\ref{enum:loopcond} no longer holds. Also notice that $g(x)$ is equivalent to swapping
function $f_{2^n-2,2^n-1}(x)$. The inverse function $f_k^{-1}$ can be easily calculated
since $f_k$ is a composition of ARX primitives.

To see $\sigma$ swap $(a,b)$, calculate the following functions:

\begin{equation*}
\setlength{\abovedisplayshortskip}{0em}
\setlength{\abovedisplayskip}{0em}
\begin{split}
f_0(a) &= 0\\
f_1(a) &= 0\\
f_i(a) &= 0\\
f_k(a) &= 2^n-2\\
g(f_k(a)) &= 2^n-1\\
\end{split}
\quad\quad
\begin{split}
f_0(b) &= b-a\\
f_1(b) &= (b-a)/2^i\\
f_i(b) &= 1\\
f_k(b) &= 2^n-1\\
g(f_k(b)) &= 2^n-2\\
\end{split}
\end{equation*}

\noindent Notice that $f_k(b)=2^n-1$ indicates the inverse function $f_k^{-1}(2^n-1)=b$,
hence $\sigma(a)=f_k^{-1}(g(f_k(a)))=b$. Similarly, there are $\sigma(b)=a$.

For $x\notin\{a,b\}$, there are $f_k(x)\notin\{2^n-2,2^n-1\}$ and thus $g(x)=x$. It
follows that $\sigma(x)=f_k^{-1}(g(f_k(x)))=f_k^{-1}(f_k(x))=x$ hence complete the proof.
\end{proof}

A Python3 implementation of the above construction is presented in Appendix
\ref{ssec:permarx}, it shows that this construction is feasible at software level.

\subsection{Completeness of \textbf{\textit{ARX}(n)}}

A more general statement regarding the expressiveness of ARX ciphers is given as:

\begin{theorem}
\label{thm:func}
For all function $f:\mathbb{Z}_{2^n}\rightarrow\mathbb{Z}_{2^n}$, $f\in\textit{ARX}(n)$.
\end{theorem}

Formally, it is said that ARX operations are functionally complete in the set of
functions over $\mathbb{Z}_{2^n}$ \cite{khovratovich2010rotational}. Khovratovich and
Nikoli{\'c} give a proof by construction for this theorem:

\begin{proof}
Given a sequence $X=x_1x_2\cdots x_n$. 

Let $s_i(X)=$
\end{proof}


\section{Instances}
\label{sec:instances}

As mentioned before, the design of ARX ciphers ensures they are efficient at both
hardware and software levels. Hardware efficiency enables ARX can be run on most
constrained platforms, who typically equips with several KBs RAM and limited CPU power.
On the other hand, being efficient at software level also accelerate the PC applications
and services, for example, easing the burden of a busy Web server who accepts intensive
TLS connections.

In this section, two instances of ARX ciphers are introduced. The first one, called
\textsc{Speck}\cite{beaulieu2015simon2}, is designated on IoT devices. The second one,
called ChaCha\cite{bernstein2008chacha}, is a general-purposed stream cipher family,
and it is recently selected as the replacement of RC4 in the
TLS\footnote{\url{https://tools.ietf.org/html/rfc7905}} standard.

\subsection{The \textsc{\textbf{Speck}} Block Cipher}

\textsc{Speck} is designed specifically for encrypting data on constrained platforms.
It is also able to offer security services on diverse platforms with different security
levels, which are controlled by two security parameters.

\textsc{Speck} block cipher adopts the notation \textsc{Speck}$2n/nm$, where $2n$ is the
block size and $mn$ is the key size (in bits). In the design, $n\in\{16,24,32,48,64\}$
and each $n$ corresponds to a set of options for $m$. The \textsc{Speck} round function
is the (Feistel-based) map
$$R_k(x,y)=((\mathrm{r}^{-\alpha}x+y)\oplus k,\ \mathrm{r}^\beta y\oplus(\mathrm{r}^{-\alpha}x+y)\oplus k)$$
where $x$ and $y$ are $n$-bits quantities, and $k$ is the round key. The rotation
parameters $\alpha$ and $\beta$, and the number of rounds are specified along with
block size $n$.

\textsc{Speck} with $m$-size words accepts the initial key $K=(\ell_{m-2},\dots,\ell_0,k_0)$,
it then generates round keys as follows:

\begin{equation*}
\setlength{\abovedisplayshortskip}{0em}
\setlength{\abovedisplayskip}{0em}
\begin{split}
\ell_{i+m-1} &= (k_i+\mathrm{r}^{-\alpha}\ell_i)\oplus i\\
k_{i+1} &= \mathrm{r}^\beta k_i \oplus \ell_{i+m-1}
\end{split}
\end{equation*}

\noindent The value $k_i$ is the $i$-th round key.

\textsc{Speck}, by designed, uses \textit{simple} round function with necessary times
of rounds for security, as it offers compact implementation hence is suited for
constrained platforms. The use of uniform parameters (across different security levels)
also approves this purpose. Moreover, \textsc{Speck} can be done entirely in-place, so
unnecessary moves of word can be avoided to obtain better performance.

\subsection{The ChaCha Stream Cipher}

The cryptosystem ChaCha is a successor of the stream cipher Salsa20
\cite{bernstein2008salsa20}, which is designed to improve diffusion at each round. By
design, ChaCha is faster than existing standard ciphers, and is aimed at users who
value speed more than secrecy. ChaCha offers 3 variants, they are differed in the number
of rounds: 8, 12 and 20 rounds versions.

ChaCha itself computes a quarter-round to update 4 32-bits state words $(a,b,c,d)$ as:

\begin{equation*}
\setlength{\abovedisplayshortskip}{0em}
\setlength{\abovedisplayskip}{0em}
\begin{split}
a &= a+b\\[1pt]
c &= c+d\\[1pt]
a &= a+b\\[1pt]
c &= c+d\\[1pt]
\end{split}
\quad
\begin{split}
d &= \mathrm{r}^{16}(d\oplus a)\\
b &= \mathrm{r}^{12}(b\oplus c)\\
d &= \mathrm{r}^{8}(d\oplus a)\\
b &= \mathrm{r}^{7}(b\oplus c)\\
\end{split}
\end{equation*}

A ChaCha quarter-round gives each input word a chance to affect each output word, hence
achieves efficient diffusion at each round: every 1-bit input difference changes 12.5
output bits on average. This is the major difference between ChaCha and its predecessor
Salsa20, who affects on 8 bits per round.

ChaCha adopts the strategy that the plaintext and ciphertext do not affect the key stream,
in other word, it is a synchronous stream cipher. To apply the stream, ChaCha simply xor
it to the plaintext, as how a pseudorandom-OTP will work.

Google proposed ChaCha20 along with Poly1305 MAC as a replacement for RC4 in TLS, to
secures TLS/SSL traffic between the Chrome browser on Android and Google's websites.
Shortly after Google, both the ChaCha20 and Poly1305 were implemented for a new
cipher in OpenSSH. These adoptions imply that the ARX cipher family does not only
essential for hardware efficiency, but also important for software performance.


\section{Conclusion}
\label{sec:conclusion}

\lipsum[1]



{\small\bibliographystyle{ieee_fullname}\bibliography{bib}}


\newpage
%\columnbreak

\section*{Appendix}

\renewcommand\thesubsection{\Alph{subsection}}

\subsection{Permutations by ARX}
\label{ssec:permarx}

The code below shows how to generate arbitrary permutation by solely ARX primitives.
The code is written in Python3 syntax, but is functionally equivalent to any other
programming languages.

\begin{figure}[h!]
\lstinputlisting[
    language=Python,
    %caption={Python program of permutations by ARX},
    label={lst:tf-rename}
]{permutation.py}
\vspace{-5mm}
\end{figure}



%\begin{figure}[t]
%\begin{center}
%\fbox{\rule{0pt}{2in} \rule{0.9\linewidth}{0pt}}
%   %\includegraphics[width=0.8\linewidth]{egfigure.eps}
%\end{center}
%   \caption{Example of caption.  It is set in Roman so that mathematics
%   (always set in Roman: $B \sin A = A \sin B$) may be included without an
%   ugly clash.}
%\label{fig:long}
%\label{fig:onecol}
%\end{figure}
%
%\begin{figure*}
%\begin{center}
%\fbox{\rule{0pt}{2in} \rule{.9\linewidth}{0pt}}
%\end{center}
%   \caption{Example of a short caption, which should be centered.}
%\label{fig:short}
%\end{figure*}


\end{document}
